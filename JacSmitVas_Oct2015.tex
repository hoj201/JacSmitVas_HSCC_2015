\documentclass{sig-alternate}
\usepackage{color}

\newtheorem{theorem}{Theorem}[section]
\newtheorem{definition}[theorem]{Definition}

\begin{document}
%
% --- Author Metadata here ---
\conferenceinfo{HSCC}{Somewhere in Nebraska}
%\CopyrightYear{2007} % Allows default copyright year (20XX) to be over-ridden - IF NEED BE.
%\crdata{0-12345-67-8/90/01}  % Allows default copyright data (0-89791-88-6/97/05) to be over-ridden - IF NEED BE.
% --- End of Author Metadata ---

\newcommand{\Traj}{Traj}
\newcommand{\BRS}{BRS}

\title{Balls}
\subtitle{[Extended Abstract]
\titlenote{A full version of this paper is available as
\textit{Author's Guide to Preparing ACM SIG Proceedings Using
\LaTeX$2_\epsilon$\ and BibTeX} at
\texttt{www.acm.org/eaddress.htm}}}
%
% You need the command \numberofauthors to handle the 'placement
% and alignment' of the authors beneath the title.
%
% For aesthetic reasons, we recommend 'three authors at a time'
% i.e. three 'name/affiliation blocks' be placed beneath the title.
%
% NOTE: You are NOT restricted in how many 'rows' of
% "name/affiliations" may appear. We just ask that you restrict
% the number of 'columns' to three.
%
% Because of the available 'opening page real-estate'
% we ask you to refrain from putting more than six authors
% (two rows with three columns) beneath the article title.
% More than six makes the first-page appear very cluttered indeed.
%
% Use the \alignauthor commands to handle the names
% and affiliations for an 'aesthetic maximum' of six authors.
% Add names, affiliations, addresses for
% the seventh etc. author(s) as the argument for the
% \additionalauthors command.
% These 'additional authors' will be output/set for you
% without further effort on your part as the last section in
% the body of your article BEFORE References or any Appendices.

\numberofauthors{3} 
\author{
% You can go ahead and credit any number of authors here,
% e.g. one 'row of three' or two rows (consisting of one row of three
% and a second row of one, two or three).
%
% The command \alignauthor (no curly braces needed) should
% precede each author name, affiliation/snail-mail address and
% e-mail address. Additionally, tag each line of
% affiliation/address with \affaddr, and tag the
% e-mail address with \email.
%
% 1st. author
\alignauthor
Ben Trovato\titlenote{Dr.~Trovato insisted his name be first.}\\
       \affaddr{Institute for Clarity in Documentation}\\
       \affaddr{1932 Wallamaloo Lane}\\
       \affaddr{Wallamaloo, New Zealand}\\
       \email{trovato@corporation.com}
% 2nd. author
\alignauthor
G.K.M. Tobin\titlenote{The secretary disavows
any knowledge of this author's actions.}\\
       \affaddr{Institute for Clarity in Documentation}\\
       \affaddr{P.O. Box 1212}\\
       \affaddr{Dublin, Ohio 43017-6221}\\
       \email{webmaster@marysville-ohio.com}
% 3rd. author
\alignauthor Lars Th{\o}rv{\"a}ld\titlenote{This author is the
one who did all the really hard work.}\\
       \affaddr{The Th{\o}rv{\"a}ld Group}\\
       \affaddr{1 Th{\o}rv{\"a}ld Circle}\\
       \affaddr{Hekla, Iceland}\\
       \email{larst@affiliation.org}
}
% There's nothing stopping you putting the seventh, eighth, etc.
% author on the opening page (as the 'third row') but we ask,
% for aesthetic reasons that you place these 'additional authors'
% in the \additional authors block, viz.
\additionalauthors{Additional authors: John Smith (The Th{\o}rv{\"a}ld Group,
email: {\texttt{jsmith@affiliation.org}}) and Julius P.~Kumquat
(The Kumquat Consortium, email: {\texttt{jpkumquat@consortium.net}}).}
\date{30 July 1999}
% Just remember to make sure that the TOTAL number of authors
% is the number that will appear on the first page PLUS the
% number that will appear in the \additionalauthors section.

\maketitle
\begin{abstract}
blah blah blah.
\end{abstract}

% A category with the (minimum) three required fields
\category{H.4}{Information Systems Applications}{Miscellaneous}
%A category including the fourth, optional field follows...
\category{D.2.8}{Software Engineering}{Metrics}[complexity measures, performance measures]

\terms{Theory}

\keywords{ACM proceedings, \LaTeX, text tagging}

\section{Introduction}
This is some intro text


\section{Problem setup}
Let us consider a control system, 
\begin{align}
	\dot{x} = f(t,x,u), \label{eq:control system}
\end{align}
 for $x$ in some metric space $X$ and $u$ in some convex space $U$.
 In order to discuss robustness for nonlinear control systems we propose the following definition:
\begin{definition}
    For a state $x_{f} \in X$, the time-$T$ \emph{backwards reachable set} is the subset
	\begin{align*}
		\BRS_{T}( x_{f} ) := \{ x_{0} \in X \mid& \exists u(\cdot) , x(\cdot) \text{ such that } \\
			&\quad \dot{x}(t) = f(t,x(t),u(t)) ,\\
			&\quad x(0) = x_{0}, x(T) = x_{T}  \}.
	\end{align*}
	For a set $\Gamma \subset X$ we define $\BRS_{T}(\Gamma) = \cup_{x \in \Gamma} \BRS_{T}(x)$.
        Given a radius $r > 0$ we say that $\Gamma$ is \emph{$r$-Robust} at time $T$ if $N_{r}( \Gamma) \subset \BRS_{T}( \Gamma )$,
	where $N_{r}( \cdot)$ refers to the $r$-neighborhood with respect to the metric on $X$.
\end{definition}

The goal of this paper is to optimize some measure of efficiency subject to a robustness constraint.
The efficiency of a trajectory $(x(t),u(t))$ is given by a user specified functional, $\mathcal{E}[x(\cdot),u(\cdot)]$.  Given an $r>0$ we could then aim to solve the optimization problem
\begin{align*}
    p^* = \sup_{u(\cdot),x(\cdot)} \mathcal{E}[ u(\cdot),x(\cdot) ]
\end{align*}
subject to the contraint that $\dot{x} = f(t,x,u)$ and that the orbit $\Gamma = x( [0,T])$ is $r$-robust.

This nonlinear infinite dimensional optimization problem is infeasible for us to solve, and so we consider a more modest problem which we can solve.
Rather than considering the space of all feasible trajectories,
we shall restrict ourselves to some finitely parametrizable subset.
In particular we will assume that there are a range of feasible trajectories $\{ \Gamma_{\alpha} \}_{\alpha \in A}$ with respect to some (tractable) indexing set $A$
which we wish to optimize over.
The efficiency, $\mathcal{E}$, can then be written as a function on $A$ given by
\begin{align*}
	E( \alpha) := \mathcal{E}( \Gamma_{\alpha}).
\end{align*}
We then seek to find the most efficient trajectory $\Gamma_{\alpha}$ subject to the constraint that the corresponding orbit, $\Gamma_{\alpha}(S^{1})$ is robust.
In summary, we desire to solve the (typically non-convex) optimization problem
\begin{align}
	p^{*} = \sup_{\alpha} E(\alpha) \label{eq:nonconvex}\\
\intertext{subject to the constraint}
	N_{r}(\Gamma_{\alpha}) \subset \BRS_{T}(\Gamma_{\alpha}). \nonumber
\end{align}
for some pre-chosen $r >0$ and $T>0$.


\section{Solution}
A reasonable solution to \eqref{eq:nonconvex} could involve convexifying this optimization problem.
This is the primary aim of this section.  Upon convexifying \eqref{eq:nonconvex} we may solve it using standard numerical convex optimization routines.

To begin, let us introduce the space of measures $\mathcal{M}(X) \equiv C(X)'$ where $C(X)$ is the space of continuous functions.

\section{Examples}
This is some Example text


\section{Conclusions}
blah blah blah

%ACKNOWLEDGMENTS are optional
\section{Acknowledgments}

%
% The following two commands are all you need in the
% initial runs of your .tex file to
% produce the bibliography for the citations in your paper.
\bibliographystyle{abbrv}
\bibliography{sigproc}  % sigproc.bib is the name of the Bibliography in this case
% You must have a proper ".bib" file
%  and remember to run:
% latex bibtex latex latex
% to resolve all references
%
% ACM needs 'a single self-contained file'!
%
%APPENDICES are optional
%\balancecolumns
%\appendix
%Appendix A
%\section{Proofs}

\end{document}
